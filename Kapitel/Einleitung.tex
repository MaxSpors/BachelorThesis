\chapter{Einleitung}


\noindent Das Standardmodell der Elementarteilchen beschreibt die fundamentalen Teilchen und Kräfte, die das Universum bestimmen. Seine Weiterentwicklung und Verfeinerung beruhen wesentlich auf dem genaueren Verständnis der Teilchen und ihrer Wechselwirkungen, die über Detektoren beobachtbar sind.\\
\\
Moderne Detektoren lassen sich in Klassen einteilen, die sich in ihren Anwendungen erheblich unterscheiden: Szintillationsdetektoren wandeln die Energie der eintreffenden ionisierenden Strahlung in Lichtpulse um, deren Messung sowohl eine genaue Bestimmung der im Detektor deponierten Energie als auch eine präzise zeitliche Zuordnung der Detektionsereignisse ermöglicht.\\
Halbleiterdetektoren wandeln die ionisierende Strahlung in Elektronen-Loch-Paare um, die zu messbaren Signalen führen. Da Halbleiter dichte Medien mit vergleichsweise geringer Ionisationsenergie sind, erlauben sie eine präzise Bestimmung der im Detektor deponierten Energie.\\
Gasdetektoren nutzen ähnliche physikalische Grundprinzipen, wobei hierbei Elektronen-Ionen-Paare erzeugt werden. Aufgrund ihrer geringeren Dichte und der Höheren Ionisationsenergie im Vergleich zu Halbleitern, ist die Energieauflösung etwas geringer. Dennoch haben sich Gasdetektoren durch ihre einfache Handhabung und Empfindlichkeit in vielen Anwendungen etabliert.  \cite{Leo}.\\
\\
Gasdetektoren sind besonders relevant, da sie in experimentellen Anordnungen aufgrund ihrer Robustheit und Flexibilität verwendet werden. Mit wachsenden Anforderungen, die für Experimente mit größerer Ereignisrate an die Detektoren gestellt werden, müssen dabei innovative Detektorkonzepte erdacht und Bestehende optimiert werden. In diese Entwicklung gliedert sich die Konzeption der Gas Electron Multiplier ein, die die Unzulänglichkeiten der zu diesem Zeitpunkt vorherrschenden Multiwire Proportional Chambers (MWPC) \cite{Sauli_Multiwire} \cite{GEM_Introduction} in Teilen kompensieren sollte. \\
Nach der Etablierung der Gas Electrom Multiplier setzte ein Optimierungsprozess ein, der sich bisher auf extensive Untersuchungen zur Detektorgeometrie \cite{BUTTNER},\cite{Bachmann}, auf die Wahl des aktiven Mediums \cite{GAS_MIX} und auf die Verbesserung der Stabilität der Detektorkonfiguration im Hinblick auf Gasentladungen \cite{Stabilitaet_Discharge} beschränkt, erste Untersuchungen für die Einstellungen  der Feldkonfiguration wurden bereits angestellt \cite{Bachmann},\cite{ottnad}, eine extensive experimentelle Untersuchung, Analyse und Optimierung der Feldeinstellungen steht dennoch aus. An diesem Punkt setzt diese Arbeit an.\\
\\
In dieser Arbeit soll die Wirkung der elektrischen Felder, die zum Betrieb des Gas Electron Multiplier verwendet werden, auf die Energieauflösung und die Signalverstärkung untersucht werden. Auf Basis dieser Ergebnisse, sollen dann Detektorkonfigurationen bestimmt werden, die es ermöglichen eine ähnliche, wenn nicht bessere Performance des Detektors bei geringerem Aufwand zu ermöglichen.\\
\\
Zu diesem Zweck wird in Kapitel II die grundlegende Funktionsweise von Gasdetektoren erörtert, wobei diese auf die zugrundeliegenden physikalischen Prinzipien reduziert werden. Damit kann dann die Spezielle Geometrie und Funktionsweise der Gas Electron Multiplier in Kapitel III skizziert werden, sodass in den Kapiteln IV und V die Wirkung des Feldes auf die Verstärkung und die Energieauflösung verstanden werden kann. Mit den Ergebnissen aus diesen Analysen lassen sich dann in Kapitel VI verschiedene interessante Detektorkonfigurationen bestimmen, die für weitere Analysen interessant sind. 
