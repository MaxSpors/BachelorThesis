\chapter{Einleitung}


\noindent Die systematische Untersuchung der Welt erfordert eine Möglichkeit individuelle Sinneseindrücke quantitativ und reproduzierbar zu erfassen und in eine verarbeitbare Form zu bringen, eine Aufgabe, die von Messgeräten erfüllt wird. Die Messgeräte, die zum Nachweis und zur Vermessung von ionisierender Strahlung oder zur Rekonstruktion von Teilchenzerfällen verwendet werden sind Detektoren.\\
\\
Moderne Detektoren können dabei in drei grobe Gattungen eingeteilt werden, die sich aufgrund der Materialien, auf denen sie basieren, in ihren Anwendungsbereichen unterscheiden: Szintillationsdetektoren wandeln Detektionsereignisse in Licht um. Sie sind besonders für Messungen geeignet, die präzise Zeitinformationen erfordern.\\
Halbleiterdetektoren nutzen Ionisationsprozesse, um Wechselwirkungen mit ionisierender Strahlung in elektronische Signale umzuwandeln. Da Halbleiter vergleichsweise dichte Medien sind und da die Ionisationsenergie in Halbleitern im Vergleich zu Gasen um den Faktor 10 geringer ist, ermöglichen sie eine sehr präzise Energiebestimmung.\\
Gasdetektoren nutzen ähnliche physikalische Grundprinzipen, sind aber einfacher zu unterhalten. Daher fallen ihre Limitationen, wie die höhere Ionisationsenergie und die geringere Dichte des Mediums deutlich schwerer ins Gewicht, da es unter Berücksichtigung des Aufwandes vergleichsweise einfach ist hinreichend gute Detektoren zu konstruieren \cite{Leo}.\\
\\
Mit den wachsenden Anforderungen, die für Experimente mit größerer Ereignisrate oder in speziellen Energieregimen an die Detektoren gestellt werden, müssen innovative Detektorkonzepte erdacht und optimiert werden. In diese Entwicklung gliedert sich die Konzeption der Gas Electron Multiplier ein, die die Unzulänglichkeiten der zu diesem Zeitpunkt vorherrschenden Multiwire Proportional Chambers (MWPC) \cite{Sauli_Multiwire} \cite{GEM_Introduction} in Teilen kompensieren sollte. Der Optimierungsprozess für Gas Electron Multiplier erstreckt sich bisher auf extensive Untersuchungen zur Detektorgeometrie \cite{BUTTNER},\cite{Bachmann}, auf die Wahl des aktiven Mediums \cite{GAS_MIX} und auf die Verbesserung der Stabilität der Detektorkonfiguration im Hinblick auf Gasentladungen \cite{Stabilitaet_Discharge}, erste Untersuchungen für die Einstellungen  der Feldkonfiguration wurden bereits angestellt \cite{Bachmann},\cite{ottnad}, eine extensive experimentelle Untersuchung, Analyse und Optimierung der Feldeinstellungen steht dennoch aus. An diesem Punkt setzt diese Arbeit an.\\
\\
In dieser Arbeit soll die Wirkung der elektrischen Felder, die zum Betrieb des Gas Electron Multiplier verwendet werden, auf die Energieauflösung und die Signalverstärkung untersucht werden. Auf Basis dieser Ergebnisse, sollen dann Detektorkonfigurationen bestimmt werden, die es ermöglichen eine ähnliche, wenn nicht bessere Performance des Detektors bei geringerem Aufwand zu ermöglichen.\\
\\
Zu diesem Zweck wird in Kapitel II die grundlegende Funktionsweise von Gasdetektoren erörtert, wobei diese auf die zugrundeliegenden physikalischen Prinzipien reduziert werden. Damit kann dann die Spezielle Geometrie und Funktionsweise der Gas Electron Multiplier in Kapitel III skizziert werden, sodass in den Kapiteln IV und V die Wirkung des Feldes auf die Verstärkung und die Energieauflösung verstanden werden kann. Mit den Ergebnissen aus diesen Analysen lassen sich dann in Kapitel VI verschiedene interessante Detektorkonfigurationen bestimmen, die für weitere Analysen interessant sind. 
