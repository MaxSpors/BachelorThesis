\chapter{Optimierung der Verstärkung eines Triple GEM Detektors}
	\section{Vorbereitung: Aufbau und Qualitätssicherung des Detektors}
	
	\section{Messmethodik}
		\subsection{Über die Eignung von $\ce{^{55}}$Fe-Isotopen}
			Die Untersuchung der Effekte der einzelnen Felder eines Triple-GEM-Detektors auf seine Energieauflösung und Verstärkung erfordert eine fundierte Kenntnis der zur Messung verwendeten Strahlung. Eine geeignete Strahlenquelle muss daher einige Anforderungen erfüllen, um den Optimierungsprozess zu ermöglichen. Die Eignung einer  $\ce{^55Fe}$ Quelle kann dabei wie folgt begründet werden:\\
			\\
			Das Isotop $\ce{^55Fe}$ zerfällt durch K-Einfang zu Mangan, welches durch Emission von Röntgenstrahlung in seinen Grundzustand übergeht, das resultierende Mangan-Isotop $\ce{^55 Mn}$ ist stabil. Mit einer Halbwertszeit von 2,75 Jahren \cite{Half_Life_FE55} gewährleistet die Quelle während des gesamten Optimierungsprozesses eine konstante Strahlleistung und so einen stabilen Energiepegel. Insbesondere sind Verzerrungen durch Sekundärzerfälle ausgeschlossen.\\
			\\
			Das Spektrum wird hauptsächlich von den $K_{\alpha}$- und $K_{\beta}$-Linien dominiert, was sicherstellt, dass die gesamte Energie im System deponiert wird (siehe Kapitel \ref{chap:Photonen
			}). Dadurch werden statistische Unsicherheiten in der Energiedeposition und damit verbundene Einflüsse auf die Energieauflösung minimiert. Eine Sammlung der relevanten Eisenlinien findet sich in Tabelle \ref{tab:Eisenlinien}.
			
			\begin{table}[h!]
				\centering
				\begin{tabular}{|l|c|c|}
					\hline
					\textbf{Linie} & \textbf{Energie / keV} & \textbf{Rel. Intensität} \\ \hline
					K\textsubscript{$\alpha$}-Photo 1    & 5,755         & 0,540           \\ \hline
					K\textsubscript{$\alpha$}-Photo 2    & 5,716         & 0,117           \\ \hline
					K\textsubscript{$\alpha$}-Photo 3    & 5,815         & 0,078           \\ \hline
					\textbf{K\textsubscript{$\alpha$}-Photo}  & 5,755         & 0,777           \\ \hline
					\textbf{K\textsubscript{$\alpha$}-Escape} & 2,892         & 0,121           \\ \hline
					K\textsubscript{$\beta$}-Photo 1     & 6,351         & 0,061           \\ \hline
					K\textsubscript{$\beta$}-Photo 2     & 6,312         & 0,013           \\ \hline
					K\textsubscript{$\beta$}-Photo 3     & 6,411         & 0,009           \\ \hline
					\textbf{K\textsubscript{$\beta$}-Photo}  & 6,351         & 0,088           \\ \hline
					\textbf{K\textsubscript{$\beta$}-Escape} & 3,488         & 0,014           \\ \hline
				\end{tabular}
				\caption{Dominierende Linien des Eisenspektrums einer $\ce{^55 Fe}$-Quelle, entnommen aus \cite{ottnad}}
				\label{tab:Eisenlinien}
			\end{table}
			
			
			\noindent Die im Optimalfall erreichbare Energieauflösung überschreitet die Energieabstände der einzelnen $K$ Linien. Diese Aussage trifft sowohl für den Photopeak, als auch für den Escape-Peak zu, sodass diese offenbar nicht eindeutig auflösbar sind. Die Bestimmung der Detektorparameter stützt sich auf die Bestimmung der Parameter des Photopeaks, es wird demnach das Spektrum angepasst, das aufgrund der Nähe der Linien vollumfänglich durch folgende Funktion beschrieben werden kann \cite{Hauer}
			 

			
			
			
			
		\subsection{Experimentelle Bestimmung der Verstärkung}
		Die Bestimmung der Verstärkung wird im folgenden auf zwei Unterschiedliche Wege durchgeführt. Die Methoden sind hierbei auf ihren jeweiligen Zweck, nämlich der quantitativeren oder qualttativeren Verstärkungsbestimmung zugeschnitten.\\
		Die qualtitative Bestimmung der Verstärkung nutzt aus, dass der MCA die Zuordnung zu seinen Bins proportional zur Amplitude des eintreffenden Signals durchführt. Nimmt man für die einzelnen Feldeinstellungen dann ein Spektrum auf, und passt eine Funktion nach Gleichung [REFERENZ] an, dann kann der Peakschwerpunkt des Photopeaks als Funktion der Felder als Proxy für die Verstärkung gewählt werden. Wandert der Schwerpunkt zu höheren Bins, sind die Signalampltiduden offenbar größer und die Signale wurden vorher stärker Verstärkt. Eine solche Methode ist offenbar nur dann sinntragend, wenn es eine Referenz gibt, zu der die Verstärkung relativ untersucht werden kann es handelt sich daher um eine qualitiative Messmethode.\\
		\\
		Quantitativ kann die Verstärkung aus einer Messung des Spektrums und einer Messung des an der Ausleseelektronik induzierten Stromes bestimmt werden. Dazu kann die Auslese des Detektors mit einem Picoamperemeter (MODELL) verbunden werden, sodass die Influenzierten Ströme gemessen werden. In einer darauffolgenden Messung muss dann in derselben Einstellung ein Spektrum gemessen werden; eine parallele Messung ist nicht möglich, da Rauscherscheinungen, die mit den Bauteilen einhergehen einen sehr erheblichen Störeinfluss implizieren, der die Messung nahezu unbrauchbar macht.\\
		Da die Linien der Eisenquelle [ABSCHNITT] hinreichend bekannt sind, können die Linien für eine Energiekalibrierung des MCA für eine gegebene Detektorkonfiguration durchgeführt werden. Damit wird das mit dem MCA gemessene Spektrum zu einem Energie-Histogramm gegebener Bin-weite. Summiert man die einzelnen Ereigniszahlen, gewichtet mit den Energien des jeweiligen Bins auf, erhält man eine Approximation für die im Detektor deponierte Energie und so, unter Berücksichtigung der mittleren Ionisationsenergie (Tabelle \ref{tab:Ionisationsenergien}) eine Näherung für den primären Ionisationsstrom, sofern man die Messdauer berücksichtigt. Die Verstärkung ergibt sich dann entsprechend über den folgenden Zusammenhang:  
		\begin{equation*}
			G=\frac{I_{\text{Readout}}}{I_{\text{Ion}}}
		\end{equation*}
		Es ist offenbar so, dass die quantitative Messung 
		\newpage
	
	\section{Optimierungsmethodik}
	
	\section{Referenzbestimmung}
	
	\section{Parameterscans für die einzelnen Felder}
		\subsection{Untersuchung der Wirkung der Felder auf die Transfereffizienzen}
		
		\subsection{Bestimmung der Auswahlintervalle}
	
	\section{Konstruktion und Überprüfung interessanter Detektorkonfigurationen}



\chapter{Optimierung der Energieauflösung eines Triple-GEM-Detektors}
	\section{Messmethodik}
	
	\section{Optimierungsmethodik}
	
	\section{Parameterscans für die einzelnen Felder}
	
	\section{Konstruktion und Überprüfung interessanter Detektorkonfigurationen}
	
	
\chapter{Bestimmung des Operations-Optimums}
	\section{Konstruktion von Feldkonfigurationen unter Berücksichtigung von Auflösung und Verstärkung}
	
	\section{Experimentelle Analyse der Konstruierten Konfigurationen}
	
\chapter{Zusammenfassung und Ausblick}
	\section{Zusammenfassung}
	
	\section{Ausblick: Optimierung auf Orts- und Zeitauflösung}		
